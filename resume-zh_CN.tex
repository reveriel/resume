% !TEX TS-program = xelatex
% !TEX encoding = UTF-8 Unicode
% !Mode:: "TeX:UTF-8"

\documentclass{resume}
\usepackage{zh_CN-Adobefonts_external} % Simplified Chinese Support using external fonts (./fonts/zh_CN-Adobe/)
% \usepackage{NotoSansSC_external}
% \usepackage{NotoSerifCJKsc_external}
% \usepackage{zh_CN-Adobefonts_internal} % Simplified Chinese Support using system fonts
\usepackage{linespacing_fix} % disable extra space before next section
\usepackage{cite}

\begin{document}
\pagenumbering{gobble} % suppress displaying page number

\name{郭兴}

\basicInfo{
  \email{guoxing@mail.ustc.edu.cn} \textperiodcentered\
  \phone{(+86) 155-5652-8177} % \textperiodcentered\
  %\linkedin[billryan8]{https://www.linkedin.com/in/billryan8}
  }

  \vspace{2mm}

\section{\faGraduationCap\  教育背景}
\datedsubsection{\textbf{中国科学技术大学}, 合肥, 安徽}{2017年 -- 至今}
\textit{在读硕士研究生}\ 计算机科学与技术, 导师: 张昱
\datedsubsection{\textbf{中国科学技术大学}, 合肥, 安徽}{2013年 -- 2017年}
\textit{学士}\ 计算机科学与技术, 计算机英才班, GPA 3.72, top10\%


\section{\faUsers\ 个人项目}

\datedsubsection{\textbf{基于$\mu$C/OS 的无人机操作系统}} {2015年 3月 -- 6月}
\begin{itemize}
  \item 操作系统实验.  https://github.com/reveriel/uCOS
  \item 负责$\mu$C/OS 到 STM 32 单片机的移植.
\end{itemize}

\datedsubsection{\textbf{流水线CPU}}{2015年}
\begin{itemize}
  % \item 组成原理实验.
  \item FPGA 编程, 实现五段流水线与通过串口的程序加载.
\end{itemize}

\datedsubsection{\textbf{RDMA加速分布式内存对象缓存系统Memcached}}{2015年6月 -- 9月}
\begin{itemize}
  \item 负责 Memcached 代码中通信代码的改写.
  \item 实现了近20倍的加速.
\end{itemize}

\datedsubsection{\textbf{基于 LLVM 的 C 语言解释器}}{2016年}
\begin{itemize}
  % \item 编译原理实验.
  \item 使用 Yacc/FLex 实现编译器前端, 后端调用 LLVM 的 JIT 引擎.
\end{itemize}

\datedsubsection{\textbf{基于 Soot 的 Android 静态分析}}{2017年}
\begin{itemize}
  % \item 本科毕业设计.
  \item 静态分析, 统计 Android 应用的库调用情况.
\end{itemize}

\datedsubsection{\textbf{基于树莓派的应用}}{2017年10月 -- 2018年4月}
\begin{itemize}
  \item Lifelog, 一个定时拍照的便携式摄像机, 智能筛选低质量照片.
  \item PiCam, 一个自动跟踪的网络摄像头, 在本地运行物体识别与物体跟踪算法.
\end{itemize}

% Reference Test
%\datedsubsection{\textbf{Paper Title\cite{zaharia2012resilient}}}{May. 2015}
%An xxx optimized for xxx\cite{verma2015large}
%\begin{itemize}
%  \item main contribution
%\end{itemize}

\section{\faCogs\ IT 技能}
% increase linespacing [parsep=0.5ex]
\begin{itemize}[parsep=0.5ex]
  \item 编程语言: C/C++/Java/Python/Lua/JavaScript, OCaml/Racket > Haskell, Coq, Ruby
  \item 平台: 2013 年开始使用 Linux/macOS
  \item 开发: Android 编程, Web 编程. LLVM.
\end{itemize}

\section{\faHeartO\ 获奖情况}
\datedline{优秀学生金奖}{2014年}
\datedline{第三届国际大学生RDMA编程竞赛(中国赛区)一等奖}{2015 年 11 月}
\datedline{第五届大学生物联网创新大赛(华东赛区)三等奖}{2018年4月}

\section{\faInfo\ 其他}
% increase linespacing [parsep=0.5ex]
\begin{itemize}[parsep=0.5ex]
  \item 技术博客: http://reveriel.com
  \item GitHub: https://github.com/reveriel
  \item 语言: 英语 - 熟练(TOEFL 96),
  \item 担任 2017 编译原理, 2018 程序设计语言基础的助教.
\end{itemize}

%% Reference
%\newpage
%\bibliographystyle{IEEETran}
%\bibliography{mycite}
\end{document}
